\documentclass[12pt,letterpaper]{article}
\usepackage{ifpdf}
\usepackage{mla}
\usepackage{comment}
\usepackage{ragged2e}
\usepackage{scrextend}
\usepackage{setspace}
\usepackage{verbatim}

\begin{document}
\begin{mla}{Matthew}{Ramirez}{Professor Ausubel}{English 1B}{\today}{The Unwinnable War}
\label{Introduction}
Drugs, they have been around since the beginning of time and have continued to be consumed regardless of the illegality of it. While the public seems to think that anyone who has or does take drugs for the pure pleasure of it must have some serious problems, rather we must look at who is in the wrong in this situation. Everyone should be given the freedom to control what goes into their body and what they do to their body. As drug use continues to prevalent in society, we must possibly look to find a way to either stop all use or legalize it. The Federal government's views on drug use needs to change to keep up with the ever changing world we live in. The use of illicit drugs in the US is becoming more and more prevalent and we need laws to reflect that. Based on a study from the National Institute on Drug Abuse, they found that illicit drug use has been increasing in the United States, from 8.3\% in 2002 to 9.4\% in 2013. Instead of simply legalizing only marijuana for recreational use, we should move towards legalizing all drugs.\\
\label{The Culture of Drugs}
While many might see that number of increasing drug use in the United States, the same study also shows that it has only really increased specifically in terms of the use of marijuana. Other drugs such as Hallucinogens, Cocaine, Prescription Drugs, and others have either stayed stagnant or in fact decreased. But why are any drugs still being used? And why are they continuing to be a part of society? Based on a study on Swiss school students done by Evangelia Liakoni et al., they found that most of the students with the average age of 17 use recreational drugs at a high rate of 27\%, while 53\% of the students claimed they used marijuana and another 17\% claimed they were using prescription drugs. When asked as to why they were taking the drugs, they responded uniformly claiming it was to be used as a relaxant to help them forget about the pressure of school and other life problems. The students used these drugs with no ill-will towards anyone else and used them at a very low controlled rate that would keep them safe from any possible health problems. In another study done by Geraint B. Osbourne and Curtis Fogel, they attempted to find the motivation as to why Canadian adults were taking recreational marijuana. Many claimed that it was to help them also unwind after a long day as well as use between friends in social settings. In a study done by Ross D. Aikins, he found that in American Higher Education the use of stimulant medications and new drugs such as opioids, cathinones, sedatives and salvia divinorum have had an increase in use. Compared to the years before from the 1960's to the 1990's, we find out that while alcohol use has gone down and marijuana use has averaged out, these new illicit drugs that make up the rest of the study have only increased to balance out the loss of other illicit drugs. So based on these studies, we can only come to the conclusion that while drug use may not have increased overall, the amount of users have continued to be the same. So with all of the campaigns against drugs, the scare tactics, one must wonder why do people continue to use them?\\
\label{The Anti Drug Culture}
The Anti Drug Campaign that has been going on in the United States has tried to show people the dangers of illicit drug use and to scare the younger children into believing these stories that were told to them. Ironically, based on a study done by Stephen Magura, he found that from 1998-2004 the National Youth Anti-Drug Media campaign in fact failed. It didn't just fail, after the campaign was over, the evaluation done after the study only proved to show that it made the youth more interested in drugs. The exact opposite outcome of the campaign happened and as a result the campaign ended, realizing they needed a new approach. While the emergence of new TV ads against the use of Drugs have been showing better results(Maria Leonora G. Comello 2013), ultimately the effect doesn't change much as those TV ads don't have the power to influence the people outright in this case. So in an effort to combat drugs of all kinds, in 1971 President Richard Nixon decided to start a War on Drugs. So what was the war about? Well President Nixon believed that in order to fight "public enemy number one", we needed to go on a full offensive. Arrests and incarcerations for the prisons became a problem as within 18 years since the start of the war, there was an increase of 126\% for drug arrests specifically towards non-white Americans(Barrile 1990). Also, although the amount of people that were being arrested were skyrocketing there was very little proof to show a decline in actual drug use. This didn't stop anything, the trafficking of drugs into the United States stayed the same and our spending continued to grow with no real end in sight (Barrile 1990). So we cannot control drug use by scare tactics in the forms of ads as well as not being able to control drug sales and trafficking, so what are we to do?\\
\label{What should we do? Pt.1 The War on Drugs}
The War on Drugs since its founding in 1971 up until now has cost the American taxpayers about \$1 trillion dollars. In 2011, the Obama administration proposed a budget of \$15.6 billion dollars for that year for the War on Drugs. Based on the Department of Homeland Security in 2009; we have seized 136,000 pounds of cocaine, 2000 pounds of heroin, 4.3 million pounds of marijuana, and 6,135 pounds of methamphetamine. It sounds as if we seized a lot in one year, but in the whole scheme of things, this has done not much to curb the consumption of the illicit drugs. The War on Drugs only continues to escalate as when the Government decides to put more and more officers and effort into stopping the trafficking, the cartels only figure out new ways to transfer the drugs to within our boarders. Cartels have been known to send more drugs than actually needed across our boarder knowing full well that not all of these drugs would make it(McCombs 2010). When they ship 100 crates for example, the expectation that maybe 30\% of them might be seized is not something they consider as "lost profit", these shipments are made so that when they are seized, their initial shipment is still unaffected. It is almost used as a way for the War on Drugs to claim they have won by capturing a lot, when in actuality nothing has changed as the amount being used has stayed constant. Not only are we losing money funding this never-ending war, we are losing countless officers lives to a never ending war. So what are we really fighting? At this point, one may wonder the reason we keep this war going on at all is the reason that we cannot admit we cannot win. As a result of this we are more content with keeping up this facade that everything is under control and we are winning.\\
\label{What should we do? Pt.2 Jail/Crime Sentencing}
As a result of the War on Drugs, our crime rate and incarceration rate has increased by 126\% as stated earlier for drug related crimes. Many people who have been convicted and jailed were minor non violent criminals who simply had been caught using or in possession of these drugs. According to the Bureau of Justice Statistics, more than half of the federal prisoners who are incarcerated are for drug related crimes in 2010. We have over 2.2 million Americans that are in prison or jail for drug related crimes. Matt Sledge of the Huffington Post estimates, "when you combine state and local spending on everything from drug-related arrests to prison, the total cost adds up to at least \$51 billion per year". Not only are we sentencing millions of people for just drug related instances alone, we are specifically targeting minorities. African Americans make up 50\% of the local and state prisoners incarcerated for drug crimes and they are 10 times more likely to be arrested for drug crimes than white kids. In a study done by the Center of Behavioral Health Statics and Quality, National Survey on Drug use and Health, they found that the African Americans used illicit drugs by adolescents were at a higher rate at 10.5\% compared to the White and Latino rate of 8.8\%. But as the White race makes up for about 60\% of the population in America, they are being jailed at a lesser rate than the other races. When the war on drugs started, it began targeting minorities, specifically the African American communities. Erik Kain, a writer for Forbes states that "African-Americans are 62 percent of drug offenders sent to state prisons, yet they represent only 12 percent of the U. S. population." Not only that, the African Americans are also charged at a rate of 13 times more than a White male(Kain 2011). So the War on Drugs has also caused us to spend more on jail, as well as convicting an entire race of people for a something that others are doing, but they are being targeted for. As a result of having this huge war, we needed to make an example out of these prisoners. Using Texas as an example, for penalty group 1 drugs such as cocaine and heroin you can receive a sentence in prison ranging from 2 years up to 99 years depending on the amount. So if someone were to sell 4 grams of cocaine, they would receive 2-20 years and a fine of \$10,000. The sentencing for voluntary manslaughter ranges from 3 to 11 years and involuntary manslaughter is from 2 to 4 years. It is extremely more and more likely to receive longer sentencing for 4 grams of cocaine than a child molester and a rapist(Sex Crimes 2016). These laws have been quite extreme for illicit drug use and sales as a result of  the War on Drugs and as a result they have continued to ruin these prisoners lives as they are being treated worse than far more horrendous crimes in the eyes of the law.\\
\label{What should we do? Pt.3 The Government regulation and control}
With the legalization of all illicit drugs, we can move forward towards government regulation and control of the market. Just as in Colorado, in terms of the sale of marijuana it was done together in coordination with the Colorado's governmental agencies to work in an effective retail system that respects the consumers as well as striving to mitigate the negatives. As Ghosh states in her article The Public Health Framework Of Legalized Marijuana In Colorado, "With coordination and direction from the Governor’s Office of Marijuana Coordination, experts from a variety of state agencies—including individuals in public and environmental health, transportation, human services (which includes child protective services and behavioral health), health care coverage and access, public safety and law enforcement, revenue, and education have been working together on marijuana-related issues." With the whole framework of the state working together to help control the whole system from top to bottom in the selling and control of marijuana, it has managed to help keep everything in working order. Colorado regulated who could sell marijuana, what times it can be purchased, and who could produce marijuana. Using the same system, we could apply this to all drugs and the FDA could help managed what level of the drugs must be to be considered pure and clean. If the government allowed production of these illicit drugs for recreational use, it would be hard for the cartels/sellers to compete. When there are legal drugs that may be slightly more expensive vs the illegal cheaper drugs, anyone would take the safer route. With the taxation of these drugs it would bring in more money for the government. A senior lecturer on economics at Harvard University Jeffrey A. Miron and a professor of economics at NYU Katherine Waldock have estimated that the legalization of drugs alone would save the government approximately \$41.3 billion dollars annually. If the drugs were sold by the government and taxed at the same rates as those of alcohol and tobacco, it would bring in revenues of \$46.7 billion dollars annually(Miron and Waldock). With the control and backing from government, it would make everything easier to track which states are using the most and coordinate more efforts such as anti-drug support systems in those areas.\\
\label{The Aftermath}
So we have solved those 3 problems, we must look at how to utilize the changes we will now face in each. With the War on Drugs ended, many non violent drug users that were incarcerated could be released and become apart of society once again. This again would free up money, as well as keeping our court rooms slightly less crowded as these crimes would not be considered illegal anymore. Now as many would argue that as the drugs have become legalized there may be more crime. Well in 2001, Portugal decriminalized the acquisition, possession, and use of small quantities of all psychoactive drugs. It did not trigger any dramatic changes in specific drug-related behavior and the number of arrests for trafficking have been brought down almost 50\% within the 1st year(Laqueur 746). As well as in Colorado, the consumption of marijuana did not increase outside of the normal highs and lows of consumption after the legalization of recreational use(Ghosh 2016). So to plays devil's advocate, lets presume we would need to increase the police force as a result of the legalization. As Miron and Waldock explained, we would save \$41.3 billion dollars annually, so using that number would could hire 150,000 new officers at a rate of \$60,000 per officer a year and still have \$32.3 billion dollars left. So we could look at those who have been jailed, as a result of them being released, we would need to way to help get the back into the workforce. With the Pell Grant expenditure cost only getting higher and higher, we could use some of the now free funds (from the legalization alone not sales), to help pay for them. According to the College Board, Pell Grants have reached a total of \$30.3 billion dollars from the 2014 to 2015 school year so with these new freed inmates, we could pay for them to have these grants even if all 2.2 million of them went to college immediately. This would cost us about \$7 billion dollars, still leaving us from our original amount of \$25.3 billion dollars. Now we can look at the government regulation and control. One of the many factors now that the government has control the the sales, taxation, and production of these drugs would be the cartel competing for sales against the now legal market. As the drugs could now be purchased safely and legally from the government, it would make anyone question to ever go to an illegal seller again when they could avoid taking the risk of being thrown in jail. The cartels itself would have trouble trying to convince its customers to want to buy from them when they could get a perfectly clean regulated drug of way better quality due to the FDA having strict guidelines in production compared to the illegal version. The government could also step in to help show how to safely use these drugs in a manner in which they show the correct doses as well as giving them ways to purchase needles among other things needed. Of course the reaction would be, why would the government try to show people how to take these drugs? Well we already have programs such as the Needle Exchange Program (NEP). In an effort to combat the use of dirty needles specifically within the drug using community to stop the spread of disease. According to Traci Green et al., more than 36 million syringes were distributed annually, mostly through large urban programs operating a stationary site. While this is not directly showing them how to use drugs, some may argue its promoting its use. If we can show that when taken at low enough dosage and it could still provide the results the user wanted, it could benefit people who may want to use them but still be able to live a normal functioning life. The government could also use that money to find a better way to start a new public initiative against drug use. In the same manner we have the campaign to stop tobacco consumption but is still legal, we can start a more reasonable campaign to stop illicit drug consumption while it is legal. If we wanted to publicly fund free rehab if you wanted to go to get clean, we could start a program where can pay for most if not all. Based on the cost of rehab for a 90 day program and at a nicer rehab center, we could afford to send 250,000 people for free yearly for a really high level of care at the cost of \$15 billion dollars. Of course this may be cheaper as the government could get contracts as well towards specific places but nonetheless it could help start a conversation to help those who are addicted now and help them get clean. Adding this to the original expenditure I've stated just from the War on Drugs savings alone, we would still have about \$10 billion dollars left that we can save. If we add the money we could possibly gain from the taxation of the drugs, we can look at a net gain of \$56.7 billion dollars. So not only can we tackle the most difficult problems such as public safety with police, public education with schooling and information, as well as paying for the people that will need help if they want to become clean, we still turn this into a net gain.\\
\label{Conclusion}
When the federal government, specifically the senators must decide the future of what we should do in a situation like this, there is no right answer for everyone. While considering many factors from the pros and cons, we must look from an unbiased point of view. When we take a step back and look at how much we have spent towards fighting this war (upwards of \$1 trillion dollars), we can easily see that we have not made much progress. Making laws more strict does not fix the problem as we have seen countless times. We tried prohibition against alcohol and that failed, we incarcerate people at insane rates for drug use assuming this would fix the problem and it also failed. There will come a time when we have to realize this is not working and we need to try something new. Legalization seems to be the only solution and with the freed up money as proven earlier, we could afford to protect ourselves from anything while having \$56.7 billion dollars left over if needed. People treat the drug problem as a criminal one, when we should consider the possibly that it may not be an issue at all. The best way we can do with the consumption of drugs is to provide a way for them to consume them in as safe of way possible by providing information and when they want to quit, a safe place for them to recover. If we just continue to jail people over and over, we never really get anywhere other than spending more and more money. To quote Vanessa Baird states in her article Legalize Drugs - All of Them, "It may sound paradoxical, but ridding ourselves of prohibition could be the best way of getting a grip."

\begin{workscited}
\bibent Aikins, Ross D. "From Recreational To Functional Drug Use: The Evolution Of Drugs In American Higher Education, 1960–2014." History Of Education 44.1 (2015): 25-43. Academic Search Complete. Web. 19 July 2016.\\
\bibent Austin, James, and Aaron David Mcvey. "Focus." The Impact of the War on Drugs (1989): n. pag. Print.\\
\bibent Carson, E. Ann, and William J. Sabol, Ph.D. "Prisoners in War." Choice Reviews Online 48.07 (2011): n. pag. U.S. Department of Justice. DOJ, Dec. 2012. Web. 20 July 2016.\\
\bibent Baird, Vanessa. "Legalize Drugs - All Of Them! (Cover Story)." New Internationalist 455 (2012): 12-17. Academic Search Complete. Web. 10 July 2016.\\
\bibent Barrile, L. G. "Book Review: Sealing the Borders: The Effects of Increased Military Participation in Drug Interdiction." Criminal Justice Review 15.1 (1990): 100-02. Web.\\
\bibent Board, The College. Trends in Student Aid 2015 | Trends in Higher Education Series – The College Board (n.d.): n. pag. College Board. College Board. Web. 21 July 2016. <http://trends.collegeboard.org/sites/default/files/trends-student-aid-web-final-508-2.pdf>.\\
\bibent Comello, Maria Leonora G. "Comparing Effects Of “My Anti-Drug” And “Above The Influence” On Campaign Evaluations And Marijuana-Related Perceptions." Health Marketing Quarterly 30.1 (2013): 35-46. Academic Search Complete. Web. 22 July 2016.\\
\bibent "Cost of Rehab - Paying for Addiction Treatment." Addiction Center Cost of Drug and Alcohol Rehab Comments. Addiction Center, n.d. Web. 21 July 2016.\\
\bibent "Drug Dealing and Drug Sales Charges." Findlaw. Find Law, n.d. Web. 20 July 2016.\\
\bibent Kain, Erik. "The War on Drugs Is a War on Minorities and the Poor." Forbes. Forbes Magazine, 28 June 2011. Web. 20 July 2016.\\
\bibent Ghaly, Sera Jane. "How Much Money Has Been Spent On The War On Drugs?" HERB. HERB, 29 Oct. 2015. Web. 20 July 2016.\\
\bibent Ghosh, Palash. "The Pros and Cons of Drug Legalization in the U.S." International Business Times. International Business Times, 19 Oct. 2010. Web. 21 July 2016.\\
\bibent Ghosh, Tista, et al. "The Public Health Framework Of Legalized Marijuana In Colorado." American Journal Of Public Health 106.1 (2016): 21-27. Academic Search Complete. Web. 23 June 2016.\\
\bibent Green, Traci C., Erika G. Martin, Sarah E. Bowman, Marita R. Mann, and Leo Beletsky. "Life After the Ban: An Assessment of US Syringe Exchange Programs' Attitudes About and Early Experiences With Federal Funding." American Public Health Association - Life After the Ban: An Assessment of US Syringe Exchange Programs’ Attitudes About and Early Experiences With Federal Funding. American Journal of Public Health, 21 Nov. 2011. Web. 22 July 2016.\\
\bibent Laqueur, Hannah. "Uses And Abuses Of Drug Decriminalization In Portugal." Law \& Social Inquiry 40.3 (2015): 746-781. Academic Search Complete. Web. 11 July 2016.\\
\bibent Liakoni, Evangelia, et al. "The Use Of Prescription Drugs, Recreational Drugs, And “Soft Enhancers” For Cognitive Enhancement Among Swiss Secondary School Students." Plos ONE 10.10 (2015): 1-12. Academic Search Complete. Web. 3 July 2016.\\
\bibent Magura, Stephen. "Failure Of Intervention Or Failure Of Evaluation: A Meta-Evaluation Of The National Youth Anti-Drug Media Campaign Evaluation." Substance Use \& Misuse 47.13/14 (2012): 1414-1420. Academic Search Complete. Web. 20 July 2016.\\
\bibent McCombs, Brady. "US Seizing Drug Money, but Cartels Have Plenty More." Arizona Daily Star. Tucson, 7 Mar. 2010. Web. 25 July 2016.\\
\bibent "Nationwide Trends." DrugFacts:. National Institute on Drug Abuse, June 2015. Web. 22 July 2016.\\
\bibent Osborne, Geraint B., and Curtis Fogel. "Understanding The Motivations For Recreational Marijuana Use Among Adult Canadians." Substance Use \& Misuse 43.3/4 (2008): 539-572. Academic Search Complete. Web. 20 July 2016.\\
\bibent "Sentencing And Punishment of Molestation." Sex Crimes. Laws, n.d. Web. 20 July 2016.\\
\bibent "Scripting Must Be Enabled to Use This Site." U.S. Customs and Border Protection. DHS, 2009. Web. 20 July 2016.\\
\bibent Sledge, Matt. "The Drug War And Mass Incarceration By The Numbers." The Huffington Post. TheHuffingtonPost.com, 4 Aug. 2013. Web. 20 July 2016.\\
\end{workscited}
\end{mla}
\end{document}
